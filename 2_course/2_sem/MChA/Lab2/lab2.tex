\documentclass[10pt]{scrartcl}
\usepackage[utf8]{inputenc}
\usepackage[english,russian]{babel}
\usepackage{indentfirst}
\usepackage{misccorr}
\usepackage{graphicx}
\usepackage{verbatim}
\usepackage[fleqn]{amsmath}
\usepackage{tikz}
\usepackage{diagbox}
\usepackage[a4paper,margin=1.0in,footskip=0.25in]{geometry}
\makeatletter
\newcommand{\verbatimfont}[1]{\renewcommand{\verbatim@font}{\ttfamily#1}}
\begin{document}
\author{Рак Алексей}
\title{Optimization_Methods/Homework1/Rak}
\begin{titlepage}
		\centering
		{\scshape\LARGE Белорусский Государственный Университет \par}
        \vfill
        {\scshape\LARGE Методы Численного Анализа\par}
        \vspace{1cm}
        {\scshape\LARGE Лабораторная работа 2\par}
        \vspace{1cm}
        {\scshape\LARGE Интерполяционный многочлен Ньютона\par}
        \vspace{2cm}
        {\LARGE Рак Алексей\par}
        \vfill
        {\large \today}
\end{titlepage}
\section*{Постановка Задачи}\noindent
Для заданной функции $f(x)$ на равномерной сетке узлов построить интерполяционный многочлен Ньютона $P_n(x)$ и вычислить:

1)$P_n(x^*), P_n(x^{**}), P_n(x^{***})$

2) Погрешность $R_n(x^{*}), R_n(x^{**}), R_n(x^{***})$

3) Абсолютные погрешности $R_{true}(x^*), R_{true}(x^{**}), R_{true}(x^{***})$

\section*{Данные}\noindent
Функция: $f(x) = 1.3 * e^x - 0.3 * sin(x)$\\
Равномерная сетка узлов: $[1, 2]$\\
Шаг $0.1$\\
$x^{*} = \frac{31}{30}, \ x^{**} = \frac{46}{30}, \ x^{***} = \frac{59}{30}$

\section*{Алгоритмы решения и формулы}\noindent
Разностными отношениями нулевого порядка называются значения функции в узлах сетки. Разностными отношениями первого порядка называются величины
$f(x_i, x_{i + 1}) = \frac{f(x_{i + 1}) - f(x_i)}{x_{i + 1} - x_i}$. Разностные отношения порядка $n + 1, \ n = 1, 2, \dots$ определяются при
помощи разностных отношений предыдущего $n$-го порядка по формуле:
\begin{gather*}
f(x_0, x_1, \dots, x_{n + 1}) = \frac{f(x_1, x_2, \dots, x_{n + 1}) - f(x_0, x_1, \dots, x_n)}{x_{n+1} - x_0}
\end{gather*}
Таблицу вида:
\begin{equation}
\begin{array}{ccccc}
	x_0 	& f(x_0) 	&				&		&  							\\
   	x_1 	& f(x_1) 	& f(x_0, x_1) 	& 		&  							\\
   	x_2 	& f(x_2) 	& f(x_1, x_2) 	& \dots	&  							\\
  	\dots 	& \dots		& \dots			& \dots	& 							\\
   	x_n 	& f(x_n) 	& f(x_{n-1},x_n)& \dots	& f(x_0, x_1, \dots, x_n)	\\
\end{array}
\end{equation}
называют таблицей разделённых разностей.\\
Будем рассматривать обычную задачу алгебраического интерполирования функции $f(x)$ по её
значениям в узлах $x_i$, $i=0, 1, \dots, n$ и пусть $P(x)$ - многочлен $n$-ой степени,
значения которого в узалх интерполирования совпадают со значениями $f(x)$. Пусть $x$ --
произвольная точка, отличная от узлов интерполирования. Запишем следующую цепочку 
тождеств:
\begin{gather*}
P(x) = P(x_0) + (x - x_0)P(x, x_0)\\
P(x, x_0) = P(x_0, x_1) + (x - x_1)P(x, x_0, x_1)\\
\dots
\end{gather*}
Эта цепочка соотношений конечна, так как разделённая разность $n + 1$-го порядка от 
многочлена $n$-ой степени равна нулю.\\
Выражая $P(x)$ через разделённые разности и учитывая то, что в узлах значения $P(x)$ и
$f(x)$ совпадают, получаем приближённую формулу:
\begin{gather*}
f(x) \approx f(x_0) + (x - x_0)f(x_0, x_1) + (x - x_0)(x - x_1)f(x_0, x_1, x_2)+\dots+
(x - x_0)(x - x_1)\dots(x - x_{n-1})f(x_0, x_1, \dots,, x_n)
\end{gather*}
Оценка погрешности:
\begin{gather*}
R_n(x) \leq w(x)P(x, x_0, \dots, x_n)
\end{gather*}
\section*{Листинг}
\verbatimfont{\small}
\begin{verbatim}
#include <vector>
#include <iomanip>
#include <cmath>
#include <iostream>

const int n = 11;
const double alpha = 1.3;

double func(double x, std::vector<std::vector<double>> dif) {
    double answer = 0;
    for (int i = 0; i < n; i++) {
        double add = dif[0][i + 1];
        for (int j = 0; j < i; j++) {
            add *= (x - dif[j][0]);
        }
        answer += add;
    }
    return answer;
}

double f(double x) {
    return (alpha * exp(x) + (1 - alpha) * sin(x));
}

void makeDif(std::vector<std::vector<double>> &dif, int _n) {
    for (int j = 2; j < _n + 1; j++) {
        for (int i = 0; i < _n - j + 1; i++) {
            dif[i][j] = (dif[i + 1][j - 1] - dif[i][j - 1]) / (dif[i + j - 1][0] - dif[i][0]);
        }
    }
}

double w(double x, std::vector<double> _x) {
    double answer = 1;
    for (int i = 0; i < n; i++) {
        answer *= (x - _x[i]);
    }
    return answer;
}

int main() {
    std::vector<double> x(n);
    x[0] = 1.0;
    for (int i = 1; i < n; i++) {
        x[i] = x[i - 1] + 0.1;
    }
    double x1 = 31.0 / 30, x2 = 46.0 / 30, x3 = 59.0 / 30;
    std::vector<std::vector<double>> dif(n + 1, std::vector<double>(n + 2));
    for (int i = 0; i < n; i++) {
        dif[i][0] = x[i];
        dif[i][1] = f(x[i]);
    }
    makeDif(dif, n);
    std::cout << "f(x1) = " << std::setprecision(15) << std::fixed << func(x1, dif) << std::endl; //
    std::cout << "f(x2) = " << std::setprecision(15) << std::fixed << func(x2, dif) << std::endl;
    std::cout << "f(x3) = " << std::setprecision(15) << std::fixed << func(x3, dif) << std::endl
              << std::endl;
    dif[n][0] = x1;
    dif[n][1] = f(x1);
    makeDif(dif, n + 1);
    std::cout << "Погрешность в х1 = " << std::setprecision(15) << std::fixed
              << dif[0][n + 1] * w(x1, x) << std::endl;
    std::cout << "Истинная погрешность в x1 = " << std::setprecision(15) << std::fixed
              << func(x1, dif) - f(x1) << std::endl << std::endl;
    dif[n][0] = x2;
    dif[n][1] = f(x2);
    makeDif(dif, n + 1);
    std::cout << "Погрешность в х2 = " << std::setprecision(15) << std::fixed
              << dif[0][n + 1] * w(x2, x) << std::endl;
    std::cout << "Истинная погрешность в x2 = " << std::setprecision(15) << std::fixed
              << func(x2, dif) - f(x2) << std::endl << std::endl;
    dif[n][0] = x3;
    dif[n][1] = f(x3);
    makeDif(dif, n + 1);
    std::cout << "Погрешность в х3 = " << std::setprecision(15) << std::fixed
              << dif[0][n + 1] * w(x3, x) << std::endl;
    std::cout << "Истинная погрешность в x3 = " << std::setprecision(15) << std::fixed
              << func(x3, dif) - f(x3) << std::endl;
    return 0;
}
\end{verbatim}
\section*{Результаты и вывод}\noindent
Выходные данные:\\
$P_n(x^*) = 3.395840703657075$\\
$P_n(x^{**}) = 5.723885855903852$\\
$P_n(x^{***})= 9.014059843851850$\\
$R_n(x^*) \leq 5.83e-13$\\
$R_n(x^{**}) \leq 7e-15$\\
$R_n(x^{***}) \leq 6.30e-13$\\
$R_{true}(x^*) = 5.84e-13$\\
$R_{true}(x^{**}) = 8e-15$\\
$R_{true}(x^{***}) = 6.29e-13$\\
Вывод:\\
Метод Ньютона нахождения интерполяционного многочлена позволяет достаточно точно интерполировать функцию, при этом абсолютная погрешность не превосходит 7е-13. Наиболее точные значения получаются вблизи центрального узла таблицы. Метод Ньютона даёт точность
примерно 2 раза больше при такой же сложности, что и метод Лагранжа.
\end{document}