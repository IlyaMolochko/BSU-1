\documentclass[10pt]{scrartcl}
\usepackage[utf8]{inputenc}
\usepackage[english,russian]{babel}
\usepackage{indentfirst}
\usepackage{misccorr}
\usepackage{graphicx}
\usepackage{verbatim}
\usepackage[fleqn]{amsmath}
\usepackage{tikz}
\usepackage{diagbox}
\usepackage[a4paper,margin=1.0in,footskip=0.25in]{geometry}
\makeatletter
\newcommand{\verbatimfont}[1]{\renewcommand{\verbatim@font}{\ttfamily#1}}
\begin{document}
\author{Рак Алексей}
\title{Optimization_Methods/Homework1/Rak}
\begin{titlepage}
		\centering
		{\scshape\LARGE Белорусский Государственный Университет \par}
        \vfill
        {\scshape\LARGE Методы Численного Анализа\par}
        \vspace{1cm}
        {\scshape\LARGE Лабораторная работа 3\par}
        \vspace{1cm}
        {\scshape\LARGE Интерполирование на равностоящих узлах\par}
        \vspace{2cm}
        {\LARGE Рак Алексей\par}
        \vfill
        {\large \today}
\end{titlepage}
\section*{Постановка Задачи}\noindent
Для заданной функции $f(x)$ на равномерной сетке узлов построить интерполяционный многочлен Ньютона $P_n(x)$ для интерполирования в начале таблицы и вычислить $x^*$, в
конце таблицы и вычислить $x^{***}$, оценить разультат, сравнить с предыдущими 
результатами и указать степень конечной разности достаточной для обеспечения точности $e^{-6}$.
\section*{Данные}\noindent
Функция: $f(x) = 1.3 * e^x - 0.3 * sin(x)$\\
Равномерная сетка узлов: $[1, 2]$\\
Шаг $0.1$\\
$x^{*} = \frac{31}{30}, \ x^{***} = \frac{59}{30}$

\section*{Алгоритмы решения и формулы}\noindent
Пусть $x$ близка к $x_0$. Положим $x = x_0 + th$. Узлы интерполирвоания расположим в следующем порядке: $x_0, x_1, \dots, x_n$:
\begin{gather*}
P_n(x) = f(x_0) + (x - x_0)f(x_0, x_1) + \dots + (x - x_0)(x - x_1)\dots(x - x_{n - 1})
f(x_0, x_1, \dots, x_n)\\
x - x_0 = th \ \ \ \ \ \ \ \ \ \ \ \ \ \ \ \ \ \ \ \ \ \ \ \ \ \ \ \ \ \ \ \ \ \ \ \ \ 
\ \ \ \ \ \ \ \ \ \ \ \ \ \ \ \ \ \ \ \ \ \ \ \ \ \ \ f(x_0) = y_0\\
x - x_1 = (x - x_0) + (x_0 - x_1) = (t - 1)h \ \ \ \ \ \ \ \ \ \ \ \ \ \ \ \ \ \ \ \ \ 
\ \ \ \ \ f(x_0, x_1) = \frac{\Delta y_0}{1!h}\\
\dots \ \ \ \ \ \ \ \ \ \ \ \ \ \ \ \ \ \ \ \ \ \ \ \ \ \ \ \ \ \ \ \ \ \ \ \ \ \ \ \ 
\ \ \ \ \ \ \ \ \ \ \ \ \ \ \ \ \ \ \ \ \ \ \ \ \ \ \ \ \ \ \ \ \ \ \ \dots\\
x - x_{n - 1} = (x - x_{n - 2}) + (x_{n - 2} - x_{n - 1}) = (t - n + 1)h \ \ \ \ \ \ \ 
f(x_0, x_1, \dots, x_n) = \frac{\Delta^n y_0}{n!h^n}
\end{gather*}
Многочлен примет вид:
\begin{gather}
P_n(x_0 + th) = y_0 + t\frac{\Delta y_0}{1!} + t(t - 1)\frac{\Delta^2 y_0}{2!} + \dots +
t(t - 1)\dots(t - n + 1)\frac{\Delta^n y_0}{n!}
\end{gather}
(1) - многочлен Ньютона для интерполирования в начале таблицы.
\begin{gather*}
R_n(x_0 + th) = \frac{f^{(n + 1)}(\xi)}{(n + 1)!}t(t - 1)\dots(t - n)h^{n + 1}
\end{gather*}
Пусть $x$ близка к $x_n$. Положим $x = x_n + th$. Узлы интерполирования расположим в 
следующем порядке: $x_n, x_{n - 1}, \dots, x_0$.
\begin{gather*}
P_n(x) = f(x_n) + (x - x_n)f(x_n, x_{n - 1}) + \dots + (x - x_n)(x - x_{n - 1})\dots(x - x_1)
f(x_n, x_{n - 1}, \dots, x_0)\\
x - x_n = th \ \ \ \ \ \ \ \ \ \ \ \ \ \ \ \ \ \ \ \ \ \ \ \ \ \ \ \ \ \ \ \ 
\ \ \ \ \ \ \ \ \ \ \ \ \ \ \ \ \ \ \ \ \ \ f(x_n) = y_n\\
x - x_{n - 1} = (x - x_n) + (x_n - x_{n - 1}) = (t + 1)h \ \ \ \ \ \ \ \ \ 
f(x_n, x_{n - 1}) = \frac{\Delta y_{n - 1}}{1!h}\\
\dots \ \ \ \ \ \ \ \ \ \ \ \ \ \ \ \ \ \ \ \ \ \ \ \ \ \ \ \ \ \ \ \ \ \ \ \ \ \ \ \ 
\ \ \ \ \ \ \ \ \ \ \ \ \ \ \ \ \ \ \ \ \ \ \ \ \ \ \dots\\
x - x_1 = (x - x_2) + x_2 - x_1) = (t + n - 1)h \ \ \ \ \ \ \ \ \ \ \ \
f(x_n, x_{n - 1}, \dots, x_0) = \frac{\Delta^n y_0}{n!h^n}
\end{gather*}
Многочлен примет вид:
\begin{gather}
P_n(x_n - th) = y_n + t\frac{\Delta y_{n - 1}}{1!} + t(t + 1)\frac{\Delta^2 y_{n - 2}}{2!} + \dots +
t(t + 1)\dots(t + n - 1)\frac{\Delta^n y_0}{n!}
\end{gather}
(2) - многочлен Ньютона для интерполирования в конце таблицы.
\begin{gather*}
R_n(x_n + th) = \frac{f^{(n + 1)}(\xi)}{(n + 1)!}t(t + 1)\dots(t + n)h^{n + 1}
\end{gather*}
\section*{Листинг}
\verbatimfont{\small}
\begin{verbatim}
#include <iostream>
#include <vector>
#include <cmath>
#include <iomanip>

const int n = 11;
const double alpha = 1.3;
const double x1 = 1.03333333333;
const double x3 = 1.96666666667;
std::vector<double> nodes;
const double x0 = 1;
const double step = 0.1;
std::vector<std::vector<double>> konechRazn;
const double accuracy = 1e-6;
const double df11max = alpha * exp(2) + (1 - alpha) * cos(2);

double f(double x) {
    return alpha * exp(x) + (1 - alpha) * sin(x);
}

static double f1(double x) {
    return alpha * exp(x) + (1 - alpha) * cos(x);
}

static double f2(double x) {
    return alpha * exp(x) - (1 - alpha) * sin(x);
}

static void makeNodes() {
    nodes.resize(n);
    for (int i = 0; i < n; i++) {
        nodes[i] = x0 + i * (2 - 1) / 10.;
    }
}

static void makeKR() {
    konechRazn.resize(n);
    for (int i = 0; i < n; i++) {
        konechRazn[i].resize(n);
        konechRazn[i][0] = f(nodes[i]);
    }
    for (int i = 1; i < n; i++) {
        for (int j = 0; j < n - i; j++) {
            konechRazn[j][i] = konechRazn[j + 1][i - 1] - konechRazn[j][i - 1];
        }
    }
}

static int factorial(int x) {
    if (x == 1 || x == 0)
        return 1;
    return x * factorial(x - 1);
}

double begin(double x) {
    double t = (x - nodes[0]) / step;
    double answer = 0, k = 1;
    for (int i = 0; i < n; i++) {
        answer += konechRazn[0][i] * k;
        k *= (t - i) / (i + 1);
    }
    return answer;
}

double end(double x) {
    double t = (x - nodes[n - 1]) / step;
    double answer = 0, k = 1;
    for (int i = 0; i < n; i++) {
        answer += konechRazn[n - 1 - i][i] * k;
        k *= (t + i) / (i + 1);
    }
    return answer;
}

int beginAndAc(double x) {
    double t = (x - nodes[0]) / step;
    double answer = 0, k = 1;
    for (int i = 0; i < n; i++) {
        answer += konechRazn[0][i] * k;
        k *= (t - i) / (i + 1);
        if (fabs(answer - f(x)) < accuracy)
            return i;
    }
    return n;
}

int endAndAc(double x) {
    double t = (x - nodes[n - 1]) / step;
    double answer = 0, k = 1;
    for (int i = 0; i < n; i++) {
        answer += konechRazn[n - 1 - i][i] * k;
        k *= (t + i) / (i + 1);
        if (fabs(answer - f(x)) < accuracy)
            return i;
    }
    return n;
}

double expectedEnd(double x) {
    double t = (x - nodes[n - 1]) / step;
    double k = 1;
    for (int i = 0; i < n; i++)
        k *= (t + i) / (i + 1);
    return pow(step, n) * k * df11max;

}

double expectedBegin(double x) {
    double t = (x - nodes[0]) / step, k = 1;
    for (int i = 0; i < n; i++)
        k *= (t - i) / (i + 1);
    return pow(step, n) * k * df11max;
}

static void print() {
    std::cout << "x*: " << std:: endl << "resalt: " << begin(x1) << std::endl;
    std::cout << "expected: " << expectedBegin(x1) << std::endl;
    std::cout << "true: " << fabs(begin(x1) - f(x1)) << std::endl;
    std::cout << "degree with <= E-6: " << beginAndAc(x1) << std::endl;
    std::cout << "x***: " << std::endl << "resalt: " << end(x3) << std::endl;
    std::cout << "expected: " << expectedEnd(x3) << std::endl;
    std::cout << "true: " << fabs(end(x3) - f(x3)) << std::endl;
    std::cout << "degree with <= E-6: " << endAndAc(x3) << std::endl;

}

int main() {
    makeNodes();
    makeKR();
    print();
    return 0;
}
\end{verbatim}
\section*{Результаты и вывод}\noindent
Выходные данные:\\
$P_n(x^*) = 3.39584$\\
$P_n(x^{***})= 9.01406$\\
$R_n(x^*) \leq 9.99631e^{-13}$\\
$R_n(x^{***}) \leq 9.99631e^{-13}$\\
$R_{true}(x^*) = 5.8753e^{-13}$\\
$R_{true}(x^{***}) = 6.18172e^{-13}$\\
Вывод:\\
Модификация многочлена Ньютона для равномерной сетки не дала выигрыша в точности: и для $x^*$, и для $x^{***}$ мы получили точности того же порядка, что и без неё. Если учитывать тот факт, что модификации проводились как для конца, так и для начала таблицы, а исходный многочлен был построен в единственном экземпляре и дал сравнимые результаты, то можно сказать, что при
необходимости нахождения максимально точного значения данная модификация не даёт достаточного выигрыша, однако она удобна, если требуется найти значения с наперёд заданной точностью.
\end{document}