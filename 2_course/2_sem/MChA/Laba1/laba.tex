\documentclass[10pt]{scrartcl}
\usepackage[utf8]{inputenc}
\usepackage[english,russian]{babel}
\usepackage{indentfirst}
\usepackage{misccorr}
\usepackage{graphicx}
\usepackage{verbatim}
\usepackage[fleqn]{amsmath}
\usepackage{tikz}
\usepackage{diagbox}
\usepackage[a4paper,margin=1.0in,footskip=0.25in]{geometry}
\makeatletter
\newcommand{\verbatimfont}[1]{\renewcommand{\verbatim@font}{\ttfamily#1}}
\begin{document}
\author{Рак Алексей}
\title{Optimization_Methods/Homework1/Rak}
\begin{titlepage}
		\centering
		{\scshape\LARGE Белорусский Государственный Университет \par}
        \vfill
        {\scshape\LARGE Методы Численного Анализа\par}
        \vspace{1cm}
        {\scshape\LARGE Лабораторная работа 1\par}
        \vspace{1cm}
        {\scshape\LARGE Интерполяционный многочлен Лагранжа\par}
        \vspace{2cm}
        {\LARGE Рак Алексей\par}
        \vfill
        {\large \today}
\end{titlepage}
\section*{Постановка Задачи}\noindent
Для заданной функции $f(x)$ на равномерной сетке узлов построить интерполяционный многочлен Лагранжа $P_n(x)$ и вычислить:

1)$P_n(x^*), P_n(x^{**}), P_n(x^{***})$

2) Погрешность $R_n(x^{*}), R_n(x^{**}), R_n(x^{***})$

3) Абсолютные погрешности $R_{true}(x^*), R_{true}(x^{**}), R_{true}(x^{***})$

\section*{Данные}\noindent
Функция: $f(x) = 1.3 * e^x - 0.3 * sin(x)$\\
Равномерная сетка узлов: $[1, 2]$\\
Шаг $0.1$\\
$x^{*} = \frac{31}{30}, \ x^{**} = \frac{46}{30}, \ x^{***} = \frac{59}{30}$

\section*{Алгоритмы решения и формулы}\noindent
Рассмотрим многочлены $l_i(x_i)$ степени $n$, удовлетворяющие условиям:
\begin{gather*}
l_i(x_i) = \sigma_{ij} = 
\begin{cases}
   1 &\text{$i = j$}\\
   0 &\text{$i \neq j$}
 \end{cases}
\end{gather*}
которые называются многочленами Лагранжа. Их можно записать в виде:
\begin{gather*}
l_i(x_i) = \sigma_{ij} = \prod_{j=0, j\neq i}^{n}\frac{x - x_j}{x_i - x_j}
\end{gather*}
Тогда интерполяционный многочлен в форме Лагранжа принимает вид: $P_n(x) = \sum_{i = 0}^n l_i(x)f_i$\\
Введём следующее обозначение: $w(x) = \prod_{j=0}^n(x - x_j)$. Легко заметить, что $\prod_{j = 0, j \neq i}^n (x_i -
x_j) = w'(x_i)$. Тогда вид многочлена Лагранжа будет следующим:
\begin{gather*}
P_n(x) = \sum_{i = 0}^{n} \frac{w(x)}{(x - x_i)w'(x_i)}f_i
\end{gather*}
Формулы оценки для погрешности:

1) $R_{true}(x) = f(x) - P_n(x)$

2) $|R_n(x)| \leq \frac{M}{(n + 1)!}|w(x)|$

3) $|P_n(x)| \leq \frac{Mh^{n + 1}}{4(n + 1)}$

\section*{Листинг}
\verbatimfont{\small}
\begin{verbatim}
#include <cmath>
#include <iomanip>
#include <iostream>

const double x0 = 1;
const double step = 0.1;
const int n = 10;
const double alpha = 1.3;
const double df11max = alpha * exp(2) - (1 - alpha) * cos(1);

double func(double x) {
    return alpha * exp(x) + (1 - alpha) * sin(x);
}

double Lagranj(double x) {
    double ans = 0;
    for (int i = 0; i <= n; i++) {
        double term = 1, xi = x0 + i * step;
        for (int j = 0; j <= n; j++) {
            if (i == j) {
                continue;
            }
            double xj = x0 + j * step;
            term = term * (x - xj) / (xi - xj);
        }
        term *= func(xi), ans += term;
    }
    return ans;
}

double calcRn(double x) {
    double ans = df11max;
    for (int i = 0; i <= n; i++)
        ans = ans * fabs(x0 + step * i - x) / (i + 1);
    return ans;
}

double calcR_true(double x) {
    return fabs(Lagranj(x) - func(x));
}

int main() {
    double x = 31.0 / 30;
    double xx = 46.0 / 30;
    double xxx = 59.0 / 30;


    std::cout << "P1 " << std::setprecision(15) << Lagranj(x) << std::endl;
    std::cout << "P2 " << std::setprecision(15) << Lagranj(xx) << std::endl;
    std::cout << "P3 " << std::setprecision(15) << Lagranj(xxx) << std::endl;
    std::cout << "R1 " << std::setprecision(15) << calcR_true(x) << std::endl;
    std::cout << "R2 " << std::setprecision(15) << calcR_true(xx) << std::endl;
    std::cout << "R3 " << std::setprecision(15) << calcR_true(xxx) << std::endl;
    std::cout << std::setprecision(15) << df11max * pow(0.1, 11) / 44 << std::endl;
    std::cout << "R1 " << std::setprecision(15) << calcRn(x) << std::endl;
    std::cout << "R2 " << std::setprecision(15) << calcRn(xx) << std::endl;
    std::cout << "R3 " << std::setprecision(15) << calcRn(xxx) << std::endl;
    return 0;
}
\end{verbatim}
\section*{Результаты и вывод}\noindent
Выходные данные:\\
$P_n(x^*) = 3.39584070365707$\\
$P_n(x^{**}) = 5.72388585590385$\\
$P_n(x^{***})= 9.01405984385184$\\
$R_n(x) \leq 2.21996900462961e-12$\\
$R_n(x^*) \leq 1.00345724778056e-12$\\
$R_n(x^{**}) \leq 9.91156290249787e-15$\\
$R_n(x^{***}) \leq 1.00345724778056e-12$\\
$R_{true}(x^*) = 5.90194559890733e-13$\\
$R_{true}(x^{**}) = 5.32907051820075e-15$\\
$R_{true}(x^{***}) = 6.21724893790088e-13$\\
Вывод:\\
Метод Лагранжа нахождения интерполяционного многочлена позволяет достаточно точно интерполировать функцию, при этом абсолютная погрешность не превосходит 3е-12. Наиболее точные значения получаются вблизи центрального узла таблицы.
\end{document}