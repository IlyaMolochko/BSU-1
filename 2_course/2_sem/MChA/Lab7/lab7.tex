\documentclass[10pt]{scrartcl}
\usepackage[utf8]{inputenc}
\usepackage[english,russian]{babel}
\usepackage{indentfirst}
\usepackage{misccorr}
\usepackage{graphicx}
\usepackage{verbatim}
\usepackage[fleqn]{amsmath}
\usepackage{tikz}
\usepackage{diagbox}
\usepackage[a4paper,margin=1.0in,footskip=0.25in]{geometry}
\makeatletter
\newcommand{\verbatimfont}[1]{\renewcommand{\verbatim@font}{\ttfamily#1}}
\begin{document}
\author{Рак Алексей}
\title{Optimization_Methods/Homework1/Rak}
\begin{titlepage}
		\centering
		{\scshape\LARGE Белорусский Государственный Университет \par}
        \vfill
        {\scshape\LARGE Методы Численного Анализа\par}
        \vspace{1cm}
        {\scshape\LARGE Лабораторная работа 7\par}
        \vspace{1cm}
        {\scshape\LARGE Метод наименьших квадратов построения элемента наилучшего приближения\par}
        \vspace{2cm}
        {\LARGE Рак Алексей\par}
        \vfill
        {\large \today}
\end{titlepage}
\section*{Постановка Задачи}\noindent
Построить элемент наилучшего приближения $\phi(x)$ степени $n = 5$ методом наименьших квадратов по
равномерной сетке узлов $x_i \in [1, 2]$, $x_i = 1 + ih$, $h = \frac{2 - 1}{m}$, $i = \overline{0, m}$,
$m = 10$ для функции $f(x) = ae^x + (1 - a)sin x$, где $a = 1.3$ используя полиномиальную аппроксимацию, т.е.
$\phi(x) = x^i, i = \overline{0, n}$. Рассчитать истинную погрешность в точках:
\begin{gather*}
x^* = x_0 + \frac{h}{3}\\
x^{**} = x_5 + \frac{h}{3}\\
x^{***} = x_{10} - \frac{h}{3}
\end{gather*}
\section*{Алгоритм решения и формулы}\noindent
1) Для определения коэффициентов полинома $\alpha_i, i = \overline{0, n}$ составим систему:
\begin{gather*}
\sum_{i = 0}^n \alpha_i (g_i, g_j) = (f, g_j), j = \overline{0, m}
\end{gather*}
где $(g_i, g_j) = \sum_{k = 0}^m x_k^{i + j}, (f, g_j) = \sum_{k = 0}^m x_k^jf(x_k)$\\
Данную систему решим методом квадратного корня, т.к. матрица системы является симметрической.\\
2) Элемент наилучшего приближения определим по формуле:\\
\begin{gather*}
\phi(x) = \sum_{i = 0}^n \alpha_i x^i
\end{gather*}
3) Посчитаем значения элемента наилучшего приближения в узлах\\
4) Для точек $x^*, x^{**}, x^{***}$ считаем истинную погрешность:
\begin{gather*}
R_{true} (x) = |f(x) - \phi(x)|
\end{gather*}
\section*{Листинг}
\verbatimfont{\small}
\begin{verbatim}
#include <cmath>
#include <iostream>
#include <vector>

double KOEF = 1.3;
int POINTS_AMOUNT = 11;
double BEGIN = 1;
double END = 2;
int POWER = 5;
double WEIGHT = 1;
double STEP = (END - BEGIN) / (POINTS_AMOUNT - 1);
std::vector<double> CHECK_POINTS{
        BEGIN + STEP * 0 + STEP / 3.0,
        BEGIN + STEP * 5 + STEP / 3.0,
        BEGIN + STEP * 10 - STEP / 3.0
};

double f(double x){
    return (KOEF * pow(exp(1), x) + (1 - KOEF) * sin(x));
}

static double q(double x, const std::vector<double>& a){
    double res = 0;
    for (int i = 0; i <= POWER; i++){

        res += pow(x, i) * a[i];

    }
    return res;
}

std::vector<std::vector<double>> straightProcess(std::vector<std::vector<double>> matrix, int n){
    double maximum;
    int ind;
    for (int i = 0; i < n - 1; i++){
        maximum = matrix[i][i];
        ind = i;
        for (int j = i + 1; j < n; j++){
            if (maximum < matrix[j][i]){
                ind = j;
                maximum = matrix[j][i];
            }
        }
        double buf;
        for (int j = 0; j <= n; j++){
            buf = matrix[i][j];
            matrix[i][j] = matrix[ind][j];
            matrix[ind][j] = buf;
        }
        for (int j = i + 1; j < n; j++){
            for (int k = i + 1; k <= n; k++){
                matrix[j][k] -= matrix[i][k] * (matrix[j][i] / matrix[i][i]);
            }
        }
    }
    return matrix;
}

std::vector<double> reversedProcess(std::vector<std::vector<double>> matrix, int n){
    std::vector<double> a(n);
    for (int i = n - 1; i > 0; i--){
        matrix[i][n] /= matrix[i][i];
        a[i] = matrix[i][n];
        for (int j = i - 1; j > -1; j--){
            matrix[j][n] -= matrix[i][n] * matrix[j][i];
        }
    }
    a[0] = matrix[0][n] / matrix[0][0];
    return a;
}

std::vector<double> gaus(std::vector<std::vector<double>> matrix, int n){
    return reversedProcess(straightProcess(matrix, n), n);
}

std::vector<double> approximation(const std::vector<double>& a){
    std::vector<double> r(3);
    for (int i = 0; i < 3; i++){

        r[i] = fabs(f(CHECK_POINTS[i]) - q(CHECK_POINTS[i], a));

    }
    return r;
}

void output(const std::vector<double>& a, const std::vector<double>& r){
    for (int i = 0; i < POINTS_AMOUNT; i++){
        std::cout << q(BEGIN + STEP * i, a) << " ";
    }
    std::cout << std::endl;
    for (int i = 0; i <= POWER; i++){
        std::cout << a[i] << " ";
    }
    std::cout << std::endl << "f:" << std::endl;
    for (int i = 0; i < 3; i++){
        std::cout << f(CHECK_POINTS[i]) << " ";
    }
    std::cout << std::endl << "q:" << std::endl;
    for (int i = 0; i < 3; i++){
        std::cout << q(CHECK_POINTS[i], a) << " ";
    }
    std::cout << std::endl;
    for (int i = 0; i < 3; i++){
        std::cout << r[i] << std::endl;
    }
}

int main(){
    std::vector<std::vector<double>> equations(POWER + 1, std::vector<double>(POWER + 2));
    for (int i = 0; i <= POWER; i++){
        for (int j = 0; j <= POWER; j++){
            for (int k = 0; k < POINTS_AMOUNT; k++){
                equations[i][j] += WEIGHT * pow((BEGIN + STEP * k), i + j);
            }
        }
        for (int k = 0; k < POINTS_AMOUNT; k++){
            equations[i][POWER + 1] += WEIGHT * f(BEGIN + STEP * k) * pow(BEGIN + STEP * k, i);
        }
    }
    std::vector<double> a = gaus(equations, POWER + 1);
    std::vector<double> r = approximation(a);
    output(a, r);
    return 0;
}
\end{verbatim}
\section*{Результаты}\noindent
Элемент наилучшего приближения:
\begin{gather*}
\phi(x) = 1.23199 + 1.28038x + 0.179989x^2 + 0.673151x^3 - 0.133061x^4 + 0.0488768x^5 
\end{gather*}
Таблица узлов:\\
\begin{table}[h]
\begin{tabular}{|c|c|}
\hline
1.0 & 3.28132 \\
\hline
1.1 & 3.63806 \\
\hline
1.2 & 4.03654\\
\hline 
1.3 & 4.48101\\
\hline 
1.4 & 4.97613\\
\hline 
1.5 & 5.52695\\
\hline 
1.6 & 6.13907\\
\hline 
1.7 & 6.81863\\
\hline 
1.8 & 7.57238\\
\hline 
1.9 & 8.40778\\
\hline 
2.0 & 9.33298\\
\hline 
\end{tabular}
\end{table}\\
\begin{table}[h]
\begin{tabular}{|c|c|c|c|}
\hline
$x$ 			& 1.0(3) 			& 1.5(3) 			& 1.9(6)  			\\
\hline
$f(x)$ 			& 3.39584 			& 5.72389 			& 9.01406 			\\
\hline 
$\phi(x)$ 		& 3.39585 			& 5.72389 			& 9.01406 			\\
\hline 
$R_{true}(x)$ 	& $4.48641e^{-6}$ 	& $3.47124e^{-6}$ 	& $4.94157e^{-6}$	\\
\hline 
\end{tabular}
\end{table}\\
\section*{Вывод}\noindent
Полученные показатели погрешности соответствуют степени построенного элемента наилучшего приближения ($n = 5$).
Однако система $g_i(x) = x^i$ не ортогональна при больших значениях  $n$ система плохо обусловлена.
Можно обойти эту трудность, строя и используя многочлены, ортогональные с заданным весом на заданной системе
точчек. В нашем же случае (при $n = 5$) обусловленность задачи удовлетворительна.
\end{document}